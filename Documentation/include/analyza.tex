\section{Analýza}

V současné době existuje mnoho různých aplikací pro psaní poznámek, přičemž
každá je zaměřena na jiný způsob práce s textem. Mezi nejpopulárnější aplikace,
se kterými jsme se setkali, patří následující programy a stránky:

\begin{itemize}
	\item \textbf{Obsidian.md} -- pokročilá aplikace pro psaní poznámek v Markdownu, zaměřená na
	      propojení jednotlivých poznámek do sítě znalostí.
	\item \textbf{Overleaf} -- online editor zaměřený na vytváření dokumentů v LaTeXu, často
	      používaný studenty technických oborů.
	\item \textbf{Microsoft Word} -- tradiční textový procesor s širokými možnostmi formátování a
	      bohatou funkcionalitou.
	\item \textbf{Notesnook} -- bezpečná a šifrovaná aplikace podporující Markdown pro psaní
	      poznámek.
\end{itemize}

Hlavním problémem při práci s těmito nástroji je jejich vzájemná
nekompatibilita. Každá aplikace využívá jiný formát pro ukládání poznámek, což
vede k potížím při sdílení a spolupráci mezi uživateli. Pokud chce například
uživatel převést markdown soubor z Obsidian.md do Microsoft Wordu, může dojít
ke ztrátě formátování nebo nekompatibilitě s určitou funkcionalitou. Uživatelé
pak často volí dvě nepohodlné alternativy: Buď se dál pokouší poznámky
konvertovat, což vede k častým chybám a ztrátě formátu textu, nebo je musí
ručně přepisovat, což je časově náročné a zbytečně odvádí pozornost od
samotného učení.

Zvažovali jsme několik možných řešení tohoto problému:

\begin{itemize}
	\item \textbf{Vytvoření konvertoru formátů} -- Mohli jsme vytvořit nástroj, který by dokázal
	      konvertovat různé formáty poznámek mezi sebou. Nicméně to by nebylo tak
	      zajímavé jako vývoj celé fullstack aplikace pro poznámky. Navíc, většina
	      konvertorů je podle nás uživatelsky nepřívětivá a často nefunguje dokonale.

	\item \textbf{Desktopová aplikace s podporou různých formátů} -- Další možností bylo vytvořit
	      nativní aplikaci, která by umožňovala práci s různými formáty a jejich snadnou
	      integraci. To znělo už přívětivěji, ale mělo to jednu zásadní nevýhodu --
	      většina uživatelů dnes pracuje na mobilních zařízeních, a vytvoření
	      multiplatformní desktopové i mobilní aplikace by vyžadovalo značné úsilí.

	\item \textbf{Webová aplikace s cloudovým úložištěm} -- Nakonec jsme se rozhodli vytvořit
	      webovou aplikaci, která poznámky ukládá přímo v cloudu a umožňuje přihlášení
	      odkudkoliv s okamžitou synchronizací změn. Díky webovým technologiím bude možné
	      v budoucnu snadno přidávat nové funkce, například podporu transpilace formátů,
	      a to buď rozšířením REST API, nebo vytvořením specializované mikroslužby, na
	      kterou by frontend mohl posílat žádosti. Aby aplikace byla plynulá a interaktivní, rozhodli
	      jsme se využít Single Page Application (SPA), což umožňuje okamžitou odezvu na
	      změny bez nutnosti neustálého načítání stránky.
\end{itemize}

\subsection{Rozdělení problému na podúlohy}

Naše zvolené řešení jsme rozdělili na několik nezávislých funkcí:

\begin{minipage}[t]{0.45\textwidth}
	\vfill
	{\large{Backend:}}

	\begin{itemize}
		\item Správa složek a podsložek
		\item Správa poznámek, včetně obsahu a umístění ve složkách
		\item Zabezpečení (uživatelské účty apod.)
		\item Exportování jako PDF, HTML nebo zdrojový kód
		\item Sdílení poznámek pomocí odkazu nebo přístupu
	\end{itemize}
	\vfill
\end{minipage}
\hfill
\begin{minipage}[t]{0.45\textwidth}
	\vfill
	{\large{Frontend:}}

	\begin{itemize}
		\item Formy pro přihlášení a registraci (správa tokenu)
		\item Panel pro správu a přehled složek (integrace s API složek)
		\item Editor poznámek
	\end{itemize}
	\vfill
\end{minipage}

Ke dni 17.03.2025 jsou hotové funkce 1-3 na backendu a 1-2 na frontendu.
