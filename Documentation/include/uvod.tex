\section{Úvod}

Každý student si během studia píše poznámky, ať už z přednášek, cvičení nebo vlastních myšlenek. Problém však nastává ve chvíli, kdy si je chce sdílet nebo doplnit od spolužáků. Každý totiž používá jiný nástroj a formát -- někdo preferuje jednoduchý plaintext, jiný používá Markdown, další spoléhá na LaTeX a někteří si vše zaznamenávají v textových dokumentech jako Microsoft Word. Tato rozmanitost formátů však komplikuje vzájemnou spolupráci a často vede ke ztrátě formátování nebo složité konverzi mezi různými systémy.

Tento problém nás přivedl k myšlence vytvořit vlastní webovou aplikaci MemoPad, která by umožnila snadné psaní, správu a sdílení poznámek v univerzálním prostředí. Hlavním cílem bylo vytvořit jednoduchý, přehledný a efektivní nástroj, který by eliminoval potíže s konverzí formátů a zajistil bezproblémovou synchronizaci mezi zařízeními. Zároveň jsme tuto práci vnímali jako výzvu -- chtěli jsme zjistit, zda dokážeme navrhnout a implementovat aplikaci, která by se mohla stát praktickým řešením tohoto problému.

Při návrhu jsme se inspirovali moderními webovými aplikacemi a rozhodli jsme se využít koncept Single Page Application (SPA) pro plynulé a rychlé uživatelské prostředí. Plánujeme také automatickou synchronizaci s cloudovým úložištěm a potenciální podporu exportu do různých formátů, aby si uživatelé mohli své poznámky snadno přizpůsobit vlastním potřebám.

Práce se skládá z několika hlavních částí. Nejprve provedeme analýzu existujících řešení a identifikujeme jejich silné a slabé stránky. Následně se zaměříme na návrh a implementaci aplikace, kde podrobně popíšeme použitou technologii jak na straně serveru (backend -- část aplikace, která běží na serveru a stará se o zpracování dat), tak na straně uživatelského rozhraní (frontend -- část aplikace, se kterou komunikuje uživatel). Poté se budeme věnovat testování funkčnosti aplikace a ověříme, že splňuje stanovené požadavky. Nakonec připravíme uživatelskou příručku, která popíše, jak aplikaci používat, a shrneme dosažené výsledky v závěru.

Tato práce nejen poskytuje ucelený pohled na vývoj webové aplikace pro poznámky, ale také otevírá možnosti jejího budoucího rozšíření o další funkce, jako je podpora formátování textu, spolupráce více uživatelů nebo integrace s dalšími cloudovými službami.
